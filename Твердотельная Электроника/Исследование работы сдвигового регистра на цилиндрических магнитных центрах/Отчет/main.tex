\documentclass[a4paper, 14pt]{extarticle}
\usepackage[dvipsnames]{xcolor}
\usepackage[top=70pt,bottom=70pt,left=48pt,right=46pt]{geometry}
\definecolor{header}{RGB}{92,184,92}
\definecolor{defenition}{RGB}{217,83,79}
\definecolor{main_title}{RGB}{66,139,202}
\definecolor{sub_header}{RGB}{91,192,222}
\usepackage[english, russian]{babel}
\usepackage[utf8]{inputenc}
\usepackage{amsmath}
\usepackage{listings}
\usepackage{subcaption}
\usepackage{graphicx}
\usepackage{amsmath}
\title{\textcolor{main_title}{Исследование работы сдвигового регистра на цилиндрических магнитных центрах}}
\author{Шмаков Владимир Евгеньевич - ФФКЭ гр. Б04-105}






\begin{document}
\maketitle

\section*{\textcolor{header}{Введение}}



Пузырьковая память, или память на цилиндрических магнитных доменах является энергонезависимой памятью, 
разработанной в Bell Labs в 1967 году \textcolor{defenition}{Эндрю Бобеком}.

Эта была одна из первых разработок в области твердотельных запоминающих устройств. 
Из-за быстроты доступа к битам, пузырьковая память может быть использована в качестве оперативной памяти.



\section*{\textcolor{header}{Цель работы}}
\begin{itemize}
    \item Ознакомиться с принципом работы сдвигового регистра на цилиндрических магнитных доменах
    \item Найти область устойчивости
\end{itemize}
\section*{\textcolor{header}{Теоретические сведения}}

\begin{figure}[htbp]
    \centering
    \begin{subfigure}{0.4\textwidth}
        \centering
        \includegraphics*[width = 0.5\linewidth]{no_field.jpg}
        \caption{Ферромагнетик в отсутствие поля}
        \label{fig:no_field}
    \end{subfigure}%
    \begin{subfigure}{0.4\textwidth}
        \centering  
        \includegraphics*[width = 0.5\linewidth]{field.png}
        \label{fig:field}
        \caption{Ферромагнетик во внешенм поле}
    \end{subfigure}
\end{figure}



\section*{\textcolor{header}{Методика}}
\subsection*{\textcolor{sub_header}{Оборудование}}

\begin{itemize}
    \item Поляризационный микроскоп
    \item Ферромагнитный образец
    \item Генераторы вращающего и сдвигающего магнитных полей
    \item Аттеньаторы
    \item Амперметры
    \item Блок питания
\end{itemize}

\subsection*{\textcolor{sub_header}{Экспериментальная установка}}



\section*{\textcolor{header}{Обработка экспериментальных данных}}

Показания амперметров линейно связаны с велинами полей $H_{rot}$ и $H_{tr}$.
Пересчитав силы тока в поля, построим область работоспособности линейного участка 
продвижения( рисунок $\ref{fig:main}$).

\begin{figure}[htbp]
    \centering
    \includegraphics*[width = 1 \textwidth]{out.png}
    \caption{Результат эксперимента.}
    \label{fig:main}
\end{figure}


\section*{\textcolor{header}{Вывод}}






\end{document}
